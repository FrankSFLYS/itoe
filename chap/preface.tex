\chapter{前\quad 言}
\par 21 世纪是信息产业飞速发展的时代。光电信息技术更是信息时代发展的关
键和核心。
\par 由光子技术和微电子技术的结合、交叉和渗透而诞生了现代光电子技术。
现代光电子技术的涵盖面特别广泛,围绕着信息的产生、传输、处理和接收过程
以及光-电、电-光相互转换过程,诞生了各种光电光电器件,如发光二极管、半导
体激光器、光电倍增管、光敏二极管、太阳能电池,以及各种光电子技术,如光纤
通信技术、节能照明技术、液晶显示技术、立体显示技术、光刻技术等,这些器件
和技术又显著地推动了人类社会生产的进步,同时也极大地提高和改善了人类
的生活质量。
\par 光电信息技术所涉及的学科知识广泛,包括光学、电学和半导体等相关基础知
识,同时它又是新材料、光子、电子、计算机等前沿学科间相互渗透、相互作用而形
成的高新技术学科。作为普通高等院校光电、电子、测控、计算机等相关专业的在
校学生,应当在大学学习期间打好基础,掌握基本光电子器件的基本原理、性质特
点以及应用发展等知识点,同时还应当具备运用所学的光电技术初步解决实际生
产生活中问题的能力。同时,要培养学生对未来光电子技术发展的求知愿望,让学
生了解日新月异的科技发展年代里出现的光电子技术热点和创新点。
\par 但是,仅仅掌握光电信息技术的基础知识还是不够的。目前,随着国际化教育
的深入发展,来华留学的学生数量和出国留学的学生数量都在逐年增加,如何让学
生在学习专员知识的同时,获得良好的英语表达能力也显得非常重要。然而,在推
行实施双语或者全英课程教学的过程中,我们常感到缺乏一本适合大学生使用的
英文版光电子技术教材,既重视光电基础知识,又注重各种基本光电器件的介绍,
同时还涵盖如何运用光电器件构建光电探测系统等内容。为此,我们着手编写了
这本英文教材。
\par 本书的内容体系沿袭了上海理工大学光电信息与计算机工程学院为光电信息
专业的学生教学所编撰的《光电信息技术》教材。编写时,采用了全英文写作,在
《光电信息技术》这本较成熟教材的基础上,从国外相关书籍和互联网上搜集并更
新了很多有益的英文资料,新增了一些当前光电领域热点技术的介绍,例如立体显
示技术、激光全息技术、光刻技术等;同时还提供了学院教师在科研工作中设计的
光电信息探测系统的教学实例。为了便于学生自学,本书中还特意增加了中英文
专业词汇生词对照表。
\par 本书的编撰得到了上海理工大学研究生创新教材建设的支持,由上海理工大
学郑继红老师主编,杨永才、贾宏志和侯文玖等几位老师参编。光电学院的部分研
究生也参与了资料的搜集和整理工作,为书稿的最终完成做了很多细致的工作,特
别是博士研究生王康妮同学,以及硕士研究生王青青、王雅楠、李道萍、高正、陈轶
阳等同学,在此一并向他们表示感谢。
\par 由于时间仓促,书中难免存在疏漏之处,恳请广大读者批评指正,以便进一步
修订改善。

\begin{flushright}
\vspace{5ex}\kaishu
    \begin{tabular}{c}
        编\hspace{1em}者 \\
        2014 年 10 月
    \end{tabular}
\end{flushright}