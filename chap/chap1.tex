\chapter{光电信息技术的物理基础}
\par 晶体管(transistor)和集成电路(integrated circuit, IC)的发展带来了许多
令人瞩目的成就。从 CD 播放器、传真机,到杂货店里的激光扫描仪,再到手机,
集成电路已经渗透到了我们日常生活的方方面面。并且半导体电子领域依然处在快速变革
当中。此外,光学也是自然科学当中的一个很美的领域,它既易懂、富有逻辑性,同时又
具有挑战性。本章将会介绍半导体物理的基础概念,包括能带、掺杂等;光学基础,包括
反射、折射;还会介绍一些电路基础知识。

\section{半导体物理基础}
\par 这一部分将会简单介绍半导体。半导体在光学中作为源和探测器都扮演着重要角色。
发光二极管(\bold{light-emitting diode}, LED)和激光二极管(\bold{laser diode}, LD)
作为光电二极管被广泛使用。电子(\bold{electron})和空穴(\bold{hole})是半导体
形成电流的主要载体(即“载流子”——译者注),它们在半导体中被能隙(\bold{energy gap})
所分隔。光子(\bold{photon})是光波能量的最小份,它们与电子的相互作用机制是光电
器件工作的关键所在。

\subsection{能带与电传导}
\par 半导体是一种导电能力介于金属(例如铜)和绝缘体(例如
玻璃)的材料。半导体是现代电子技术的基础,包括晶体管(\bold{transistor})、太阳
能电池(\bold{solar cell})、发光二极管、量子点(\bold{quantum dots})以及数字
和模拟集成电路在内都有半导体的影子。
\par 半导体有非常多特殊的特性,其中之一是通过加入杂质(称为“掺杂”,\bold{doping})
或者与另外的现象相互作用,例如电场或光照,可以使半导体的电导率会发生改变,这一
能力使得半导体在组成放大、开关或转换输入能量器件时非常有用。关于半导体这一特性
的现代理解基于半导体物理对于原子晶格内电子的移动的解释。
\par 半导体的定义源于它的电导率介于金属和绝缘体之间的独特性质。这些材料之间的
区别可以从电子的量子态来理解,每个量子态都包括零个或一个电子(遵循泡利不相容原
理,\bold{Pauli exclusion principle}),这些状态的形成与材料中电子的带状结构
有关。电子具有电导率是因为量子态中的电子出现了移位现象(在材料中移动),但是为了
传输电子,量子态必须是部分填充的,也就是说只有部分时间被一个电子占据。如果一个量
子态总是被一个电子占据,那么它就处在非自由态,会阻止其他电子运动通过这一个量子态。
这些量子态的能量很重要,因为只有能量处在费米能级(\bold{Fermi level})附近的量子
态才会被部分填充。
\par 如果一个材料中部分填充的量子态很多,并且都有移位,那么这种材料就拥有高电导率。
金属拥有许多能量处在费米能级附近且部分填充的量子态,所以金属是电的良导体,与之相反
的是绝缘体,只有很少一部分量子态部分填充,并且它们的费米能级处于带隙之中,几乎没有
能态能够处在费米能级附近。但是需要注意的是,绝缘体的电导率可以通过升高温度来提升,
因为温度可以提供能量给电子来促使它们穿过带隙,其中包括在带隙下方的部分填充态(价带,
\bold{valence band})和带隙上方的部分填充态(导带,\bold{conduction band})。
本征半导体(\bold{intrinsic semiconductor})的带隙要比绝缘体的带隙要窄,在室温条件
下就有大量电子可以被激发穿过带隙。
\par 纯净(\bold{pure})的半导体物质不那么有用,因为它既不是好的绝缘体又不是好的导体。
但是,半导体(也包括部分绝缘体,称为“半绝缘体”)的一个重要的特性是它们的电导率可以通
过杂质的掺杂浓度或外加电场的通断来升高和控制。通过掺杂或加电场,半导体的导带和价带都
会更加靠近费米能级,因此部分填充的量子态数目就会显著增加。
\par 有些半导体材料拥有较宽的带隙,有时也视作半绝缘体。

\section{光学基础}
\section{电路基础}
\section*{参考文献}\addcontentsline{toc}{section}{参考文献}
\section*{习题}\addcontentsline{toc}{section}{题目}